\chapter{Aktueller Forschungsstand}
Die Verwendung eines Rollstuhls als Eingabegerät, ist in der Forschung vor allem in Bereichen zu finden, welche sich mit der Inklusion körperlich beeinträchtigter Menschen befassen.
Dabei werden oftmals aufwändige Konstruktionen um einen Rollstuhl herum benötigt, um die Rotationen der Räder zu messen und gleichzeitig mithilfe von Motoren verschiedene Untergründe und Bewegungen zu simulieren.

So erschien beispielsweise im Jahr 2011 das Paper: \textit{A full motion manual \\wheelchair simulator for rehabilitation research} von Crichlow et al.\cite{crichlowFullMotionManual2011}.
In diesem wird eine Rollstuhl-Simulation vorgestellt, bei dem ein Rollstuhl auf einer \textit{Motion-Platform} steht.
Der Nutzer kann dann, umgeben von einem 240° Bildschirm, durch eine virtuelle Umgebung fahren.
Dabei können unebene Bewegungen, wie „\textit{tip-overs}“ und „\textit{wheelies}“ simuliert werden, um alltägliche Situationen zu üben\cite{crichlowFullMotionManual2011}.
Dazu werden die Räder zusätzlich zum Nutzer von Motoren gedreht, um Krafteinwirkungen zu simulieren.
Jedoch werden ausschließlich Bewegungen im virtuellen Raum simuliert, sodass keine Interaktionen mit der Umgebung stattfinden können.
Zudem ist die Konstruktion entsprechend groß, unpraktikabel und teuer.
Für private Kunden ist eine so aufwändige Konstruktion unattraktiv.

Im Jahr 2000 erschien das Paper: \textit{Simulation of the behaviour of a powered wheelchair using virtual reality}\cite{ninissSimulationBehaviourPowered2000}.
In diesem wird von Niniss und Nadif beschrieben, wie ein elektrischer Rollstuhl in einer virtuellen Welt simuliert werden kann.
Jedoch beschränkt sich die Eingabe auf einen Thumbstick, der häufig an elektrischen Rollstühlen zu finden ist.
In Ansätzen wird schon hier beschrieben, welche Auswirkungen die Rotation der Räder des Rollstuhls, auf dessen Bewegung haben.
Jedoch beschränkt sich die Simulation auf ideale Bewegungen und es gibt keine Interaktionsmöglichkeiten neben der Bewegung im Raum.

Für private Kunden, die ohne große Konstruktion, wie bei Crichlow, ihren Rollstuhl als Eingabegerät an ihrem Heim-Computer nutzen wollen, sind die bisher entwickelten Systeme ungeeignet.
Es bedarf eines kleineren und kostengünstigeren Systems, dass zur Steuerung von herkömmlicher Software genutzt werden kann.
Da es sich um einen nicht-elektrischen Rollstuhl handelt, wird eine komplexere Modellierung des Rollstuhls benötigt, sodass die Analysen von Niniss und Nadif erweitert werden müssen.