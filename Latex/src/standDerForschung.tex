\chapter{Stand der Forschung}
Das Verwenden eines Rollstuhls als Eingabegerät ist zu großen Teilen in der Forschung zu finden, die sich damit beschäftigt, beeinträchtigten Menschen zu helfen.
So gibt es bislang Studien, die sich mit der Frage beschäftigen, inwieweit Virtual Reality genutzt werden kann, um Menschen, die seit kurzem einen Rollstuhl verwenden müssen, bei der Eingewöhnung zu helfen\cite{arlatiVirtualRealitybasedWheelchair2020}.
Jedoch ist in diesen Fällen der Rollstuhl meist nur ein passives Eingabegerät.
Der Rollstuhl ist nicht aufgebockt, sondern die Bewegung im virtuellen Raum findet durch das Bewegen der VR-Brille statt.
Dadurch wird eine Rollstuhl-Simulation gespart, jedoch ist der Nutzer in der Fortbewegung im virtuellen Raum begrenzt, durch die physischen Begrenzungen im echten Raum.

Das im Jahr \citeyear{harrisionDevelopmentWheelchairVirtual2000} erschiene Paper: \textit{Development of a wheelchair virtual reality platform for use in evaluating wheelchair access} geht hingegen der Frage nach, inwieweit ein realer Rollstuhl im Virtuellen simuliert werden kann\cite{harrisionDevelopmentWheelchairVirtual2000}.
Dazu wurde eine „\textit{Motion-Platform}“ konstruiert, auf den ein handelsüblicher Rollstuhl platziert werden kann.
Die Konstruktion simuliert über die reine Fortbewegung hinaus, auch die Krafteinwirkung bei Schrägen und unebenen Böden.
Im Paper wird jedoch nicht ausführlich auf die hard- oder softwareseitige Modellierung des virtuellen Rollstuhls oder der Plattform eingegangen.
Zudem ist die Konstruktion entsprechend groß, unpraktikabel und teuer.
Für mögliche Endkunden würde eine so aufwändige Konstruktion einen Sinn ergeben.

Ebenfalls im Jahr \citeyear{ninissSimulationBehaviourPowered2000} erschien das Paper: \textit{Simulation of the behaviour of a powered wheelchair using virtual reality}\cite{ninissSimulationBehaviourPowered2000}.
In diesem wird beschrieben, wie ein elektrischer Rollstuhl in der virtuellen Realität simuliert werden kann.
Jedoch beschränkt sich die Eingabe auf einen Thumbstick, der häufig an elektrischen Rollstühlen zu finden ist.
In Ansätzen wird schon hier beschrieben, welche Auswirkungen die Rotation der Räder des Rollstuhls, auf dessen Rotation haben.
Jedoch beschränkt sich die Simulation auf ideale Bewegungen und es gibt keine weiteren Interaktionsmöglichkeiten außer die Bewegung im Raum.

Diese Arbeit geht einen Schritt weiter und versucht ein System zu entwickeln, bei dem ein Rollstuhl im virtuellen Raum simuliert werden kann.
Dabei soll, anders als bei Harrision und Co., auf aufwändige Konstruktionen verzichtet werden, sodass das System kostengünstig und praktikabel für Endkunden ist.
Trotzdem soll der simulierte Rollstuhl, im Gegensatz zur Arbeit von Niniss und Nadif, eine realistische Simulation der Räder und möglichst viele Interaktionen über die reine Fortbewegung hinaus beinhalten.