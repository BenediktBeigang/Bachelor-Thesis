\begin{figure}[h]
    \centering
    \pgfplotstableread{./plots/gyroMax.dat}{\data}
    \begin{tikzpicture}
        \begin{axis}[
            xlabel = {$\mathrm{Zeit}\ [\si{\second}]$},
            ylabel = {$\mathrm{Winkelgeschwindigkeit}\ [\si{\degree/\second}]$},
            xmin = 19, xmax = 92,
            ymin = -1000, ymax = 1000,
            xtick distance = 10,
            ytick distance = 250,
            grid = both,
            minor tick num = 1,
            major grid style = {lightgray},
            minor grid style = {lightgray!25},
            width = 0.95\textwidth,
            height = 0.75\textwidth,
            legend cell align = {left},
            legend pos = south west
            ]

            \addplot[blue, mark = none] table [x = {time}, y = {left}] {\data};
            \addplot[orange, mark = none] table [x = {time}, y = {right}] {\data};

            \addplot[black, mark = none, domain = 19:92, samples = 2, dashed] {792};
            \addplot[black, mark = none, domain = 19:92, samples = 2, dashed, forget plot] {-892};

            \legend{
                $\omega_L$,
                $\omega_R$,
                Maximaler Wert
            }
        \end{axis}
    \end{tikzpicture}
    \caption{Winkelgeschwindigkeiten der Räder im Verlauf der Zeit, bei dem die Testperson ein Rad, mit der dominanten Hand, so schnell wie möglich dreht}
    \label{fig:gyroMax}
\end{figure}