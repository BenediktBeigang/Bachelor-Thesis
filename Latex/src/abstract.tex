\chapter*{Abstract}
In dieser Arbeit wird ein System vorgestellt, welches ermöglicht einen Rollstuhl als Eingabegerät in Software zu verweden.
Dabei ist es dem Nutzer möglich durch einen virtuellen Raum zu navigieren.
Es wurde zunächst ein eingebettetes System entwickelt und getestet, dass die Rotation der Räder misst und an eine weitere Software sendet.
Diese Software bildet die Rotationsdaten in Eingaben eines Spielcontrollers ab.
Die entwickelte Abbildung ermöglicht dem Nutzer in einem virtuellen dreidimensionalen Raum sich frei umzuschauen, sich auf einer Ebene fortzubewegen und vier Aktionstasten des Spielcontrollers zu betätigen.
Dabei werden die gewünschten Eingaben des Nutzers schnell und präzise verarbeitet.